% Options for packages loaded elsewhere
\PassOptionsToPackage{unicode}{hyperref}
\PassOptionsToPackage{hyphens}{url}
%
\documentclass[
]{article}
\usepackage{amsmath,amssymb}
\usepackage{iftex}
\ifPDFTeX
  \usepackage[T1]{fontenc}
  \usepackage[utf8]{inputenc}
  \usepackage{textcomp} % provide euro and other symbols
\else % if luatex or xetex
  \usepackage{unicode-math} % this also loads fontspec
  \defaultfontfeatures{Scale=MatchLowercase}
  \defaultfontfeatures[\rmfamily]{Ligatures=TeX,Scale=1}
\fi
\usepackage{lmodern}
\ifPDFTeX\else
  % xetex/luatex font selection
\fi
% Use upquote if available, for straight quotes in verbatim environments
\IfFileExists{upquote.sty}{\usepackage{upquote}}{}
\IfFileExists{microtype.sty}{% use microtype if available
  \usepackage[]{microtype}
  \UseMicrotypeSet[protrusion]{basicmath} % disable protrusion for tt fonts
}{}
\makeatletter
\@ifundefined{KOMAClassName}{% if non-KOMA class
  \IfFileExists{parskip.sty}{%
    \usepackage{parskip}
  }{% else
    \setlength{\parindent}{0pt}
    \setlength{\parskip}{6pt plus 2pt minus 1pt}}
}{% if KOMA class
  \KOMAoptions{parskip=half}}
\makeatother
\usepackage{xcolor}
\usepackage[margin=1in]{geometry}
\usepackage{color}
\usepackage{fancyvrb}
\newcommand{\VerbBar}{|}
\newcommand{\VERB}{\Verb[commandchars=\\\{\}]}
\DefineVerbatimEnvironment{Highlighting}{Verbatim}{commandchars=\\\{\}}
% Add ',fontsize=\small' for more characters per line
\usepackage{framed}
\definecolor{shadecolor}{RGB}{248,248,248}
\newenvironment{Shaded}{\begin{snugshade}}{\end{snugshade}}
\newcommand{\AlertTok}[1]{\textcolor[rgb]{0.94,0.16,0.16}{#1}}
\newcommand{\AnnotationTok}[1]{\textcolor[rgb]{0.56,0.35,0.01}{\textbf{\textit{#1}}}}
\newcommand{\AttributeTok}[1]{\textcolor[rgb]{0.13,0.29,0.53}{#1}}
\newcommand{\BaseNTok}[1]{\textcolor[rgb]{0.00,0.00,0.81}{#1}}
\newcommand{\BuiltInTok}[1]{#1}
\newcommand{\CharTok}[1]{\textcolor[rgb]{0.31,0.60,0.02}{#1}}
\newcommand{\CommentTok}[1]{\textcolor[rgb]{0.56,0.35,0.01}{\textit{#1}}}
\newcommand{\CommentVarTok}[1]{\textcolor[rgb]{0.56,0.35,0.01}{\textbf{\textit{#1}}}}
\newcommand{\ConstantTok}[1]{\textcolor[rgb]{0.56,0.35,0.01}{#1}}
\newcommand{\ControlFlowTok}[1]{\textcolor[rgb]{0.13,0.29,0.53}{\textbf{#1}}}
\newcommand{\DataTypeTok}[1]{\textcolor[rgb]{0.13,0.29,0.53}{#1}}
\newcommand{\DecValTok}[1]{\textcolor[rgb]{0.00,0.00,0.81}{#1}}
\newcommand{\DocumentationTok}[1]{\textcolor[rgb]{0.56,0.35,0.01}{\textbf{\textit{#1}}}}
\newcommand{\ErrorTok}[1]{\textcolor[rgb]{0.64,0.00,0.00}{\textbf{#1}}}
\newcommand{\ExtensionTok}[1]{#1}
\newcommand{\FloatTok}[1]{\textcolor[rgb]{0.00,0.00,0.81}{#1}}
\newcommand{\FunctionTok}[1]{\textcolor[rgb]{0.13,0.29,0.53}{\textbf{#1}}}
\newcommand{\ImportTok}[1]{#1}
\newcommand{\InformationTok}[1]{\textcolor[rgb]{0.56,0.35,0.01}{\textbf{\textit{#1}}}}
\newcommand{\KeywordTok}[1]{\textcolor[rgb]{0.13,0.29,0.53}{\textbf{#1}}}
\newcommand{\NormalTok}[1]{#1}
\newcommand{\OperatorTok}[1]{\textcolor[rgb]{0.81,0.36,0.00}{\textbf{#1}}}
\newcommand{\OtherTok}[1]{\textcolor[rgb]{0.56,0.35,0.01}{#1}}
\newcommand{\PreprocessorTok}[1]{\textcolor[rgb]{0.56,0.35,0.01}{\textit{#1}}}
\newcommand{\RegionMarkerTok}[1]{#1}
\newcommand{\SpecialCharTok}[1]{\textcolor[rgb]{0.81,0.36,0.00}{\textbf{#1}}}
\newcommand{\SpecialStringTok}[1]{\textcolor[rgb]{0.31,0.60,0.02}{#1}}
\newcommand{\StringTok}[1]{\textcolor[rgb]{0.31,0.60,0.02}{#1}}
\newcommand{\VariableTok}[1]{\textcolor[rgb]{0.00,0.00,0.00}{#1}}
\newcommand{\VerbatimStringTok}[1]{\textcolor[rgb]{0.31,0.60,0.02}{#1}}
\newcommand{\WarningTok}[1]{\textcolor[rgb]{0.56,0.35,0.01}{\textbf{\textit{#1}}}}
\usepackage{graphicx}
\makeatletter
\def\maxwidth{\ifdim\Gin@nat@width>\linewidth\linewidth\else\Gin@nat@width\fi}
\def\maxheight{\ifdim\Gin@nat@height>\textheight\textheight\else\Gin@nat@height\fi}
\makeatother
% Scale images if necessary, so that they will not overflow the page
% margins by default, and it is still possible to overwrite the defaults
% using explicit options in \includegraphics[width, height, ...]{}
\setkeys{Gin}{width=\maxwidth,height=\maxheight,keepaspectratio}
% Set default figure placement to htbp
\makeatletter
\def\fps@figure{htbp}
\makeatother
\setlength{\emergencystretch}{3em} % prevent overfull lines
\providecommand{\tightlist}{%
  \setlength{\itemsep}{0pt}\setlength{\parskip}{0pt}}
\setcounter{secnumdepth}{-\maxdimen} % remove section numbering
\ifLuaTeX
  \usepackage{selnolig}  % disable illegal ligatures
\fi
\IfFileExists{bookmark.sty}{\usepackage{bookmark}}{\usepackage{hyperref}}
\IfFileExists{xurl.sty}{\usepackage{xurl}}{} % add URL line breaks if available
\urlstyle{same}
\hypersetup{
  pdftitle={R Notebook},
  hidelinks,
  pdfcreator={LaTeX via pandoc}}

\title{R Notebook}
\author{}
\date{\vspace{-2.5em}}

\begin{document}
\maketitle

\begin{Shaded}
\begin{Highlighting}[]
\CommentTok{\# Loading in packages}

\FunctionTok{library}\NormalTok{(readr)}
\FunctionTok{library}\NormalTok{(dplyr)}
\end{Highlighting}
\end{Shaded}

\begin{verbatim}
## 
## Attaching package: 'dplyr'
\end{verbatim}

\begin{verbatim}
## The following objects are masked from 'package:stats':
## 
##     filter, lag
\end{verbatim}

\begin{verbatim}
## The following objects are masked from 'package:base':
## 
##     intersect, setdiff, setequal, union
\end{verbatim}

\begin{Shaded}
\begin{Highlighting}[]
\FunctionTok{library}\NormalTok{(ggplot2)}
\end{Highlighting}
\end{Shaded}

\begin{enumerate}
\def\labelenumi{\arabic{enumi}.}
\setcounter{enumi}{1}
\tightlist
\item
  The data set
\end{enumerate}

The dataset we will use contains one week of data from a sample of
players who played Candy Crush back in 2014. The data is also from a
single episode, that is, a set of 15 levels. It has the following
columns:

\begin{verbatim}
player_id: a unique player id
dt: the date
level: the level number within the episode, from 1 to 15.
num_attempts: number of level attempts for the player on that level and date.
num_success: number of level attempts that resulted in a success/win for the player on that level and date.
\end{verbatim}

The granularity of the dataset is player, date, and level. That is,
there is a row for every player, day, and level recording the total
number of attempts and how many of those resulted in a win.

Now, let's load in the dataset and take a look at the first couple of
rows.

\begin{Shaded}
\begin{Highlighting}[]
\CommentTok{\# Reading in the data}
\NormalTok{data }\OtherTok{\textless{}{-}} \FunctionTok{read\_csv}\NormalTok{(}\StringTok{"datasets/candy\_crush.csv"}\NormalTok{)}
\end{Highlighting}
\end{Shaded}

\begin{verbatim}
## Rows: 16865 Columns: 5
## -- Column specification --------------------------------------------------------
## Delimiter: ","
## chr  (1): player_id
## dbl  (3): level, num_attempts, num_success
## date (1): dt
## 
## i Use `spec()` to retrieve the full column specification for this data.
## i Specify the column types or set `show_col_types = FALSE` to quiet this message.
\end{verbatim}

\begin{Shaded}
\begin{Highlighting}[]
\CommentTok{\# Printing out the first six rows}
\FunctionTok{head}\NormalTok{(data)}
\end{Highlighting}
\end{Shaded}

\begin{verbatim}
## # A tibble: 6 x 5
##   player_id                        dt         level num_attempts num_success
##   <chr>                            <date>     <dbl>        <dbl>       <dbl>
## 1 6dd5af4c7228fa353d505767143f5815 2014-01-04     4            3           1
## 2 c7ec97c39349ab7e4d39b4f74062ec13 2014-01-01     8            4           1
## 3 c7ec97c39349ab7e4d39b4f74062ec13 2014-01-05    12            6           0
## 4 a32c5e9700ed356dc8dd5bb3230c5227 2014-01-03    11            1           1
## 5 a32c5e9700ed356dc8dd5bb3230c5227 2014-01-07    15            6           0
## 6 b94d403ac4edf639442f93eeffdc7d92 2014-01-01     8            8           1
\end{verbatim}

\begin{enumerate}
\def\labelenumi{\arabic{enumi}.}
\setcounter{enumi}{2}
\tightlist
\item
  Checking the data set
\end{enumerate}

Now that we have loaded the dataset let's count how many players we have
in the sample and how many days worth of data we have.

\begin{Shaded}
\begin{Highlighting}[]
\CommentTok{\# Count and display the number of unique players}
\FunctionTok{print}\NormalTok{(}\StringTok{"Number of players:"}\NormalTok{)}
\end{Highlighting}
\end{Shaded}

\begin{verbatim}
## [1] "Number of players:"
\end{verbatim}

\begin{Shaded}
\begin{Highlighting}[]
\NormalTok{data}\SpecialCharTok{$}\NormalTok{player\_id }\SpecialCharTok{\%\textgreater{}\%} \FunctionTok{unique}\NormalTok{() }\SpecialCharTok{\%\textgreater{}\%} \FunctionTok{length}\NormalTok{()}
\end{Highlighting}
\end{Shaded}

\begin{verbatim}
## [1] 6814
\end{verbatim}

\begin{Shaded}
\begin{Highlighting}[]
\CommentTok{\# Display the date range of the data}
\FunctionTok{print}\NormalTok{(}\StringTok{"Period for which we have data:"}\NormalTok{)}
\end{Highlighting}
\end{Shaded}

\begin{verbatim}
## [1] "Period for which we have data:"
\end{verbatim}

\begin{Shaded}
\begin{Highlighting}[]
\FunctionTok{range}\NormalTok{(data}\SpecialCharTok{$}\NormalTok{dt)}
\end{Highlighting}
\end{Shaded}

\begin{verbatim}
## [1] "2014-01-01" "2014-01-07"
\end{verbatim}

\begin{enumerate}
\def\labelenumi{\arabic{enumi}.}
\setcounter{enumi}{3}
\tightlist
\item
  Computing level difficulty
\end{enumerate}

Within each Candy Crush episode, there is a mix of easier and tougher
levels. Luck and individual skill make the number of attempts required
to pass a level different from player to player. The assumption is that
difficult levels require more attempts on average than easier ones. That
is, the harder a level is, the lower the probability to pass that level
in a single attempt is.

A simple approach to model this probability is as a Bernoulli process

; as a binary outcome (you either win or lose) characterized by a single
parameter pwin: the probability of winning the level in a single
attempt. This probability can be estimated for each level as:

For example, let's say a level has been played 10 times and 2 of those
attempts ended up in a victory. Then the probability of winning in a
single attempt would be pwin = 2 / 10 = 20\%.

Now, let's compute the difficulty pwin separately for each of the 15
levels.

\begin{Shaded}
\begin{Highlighting}[]
\CommentTok{\# Calculating level difficulty}
\NormalTok{difficulty }\OtherTok{\textless{}{-}}\NormalTok{ data }\SpecialCharTok{\%\textgreater{}\%}
 \FunctionTok{group\_by}\NormalTok{(level) }\SpecialCharTok{\%\textgreater{}\%}
 \FunctionTok{summarise}\NormalTok{(}\AttributeTok{attempts =} \FunctionTok{sum}\NormalTok{(num\_attempts), }\AttributeTok{wins =} \FunctionTok{sum}\NormalTok{(num\_success)) }\SpecialCharTok{\%\textgreater{}\%}
 \FunctionTok{mutate}\NormalTok{(}\AttributeTok{p\_win =}\NormalTok{ wins }\SpecialCharTok{/}\NormalTok{ attempts)}

\CommentTok{\# Printing out the level difficulty}
\NormalTok{difficulty}
\end{Highlighting}
\end{Shaded}

\begin{verbatim}
## # A tibble: 15 x 4
##    level attempts  wins  p_win
##    <dbl>    <dbl> <dbl>  <dbl>
##  1     1     1322   818 0.619 
##  2     2     1285   666 0.518 
##  3     3     1546   662 0.428 
##  4     4     1893   705 0.372 
##  5     5     6937   634 0.0914
##  6     6     1591   668 0.420 
##  7     7     4526   614 0.136 
##  8     8    15816   641 0.0405
##  9     9     8241   670 0.0813
## 10    10     3282   617 0.188 
## 11    11     5575   603 0.108 
## 12    12     6868   659 0.0960
## 13    13     1327   686 0.517 
## 14    14     2772   777 0.280 
## 15    15    30374  1157 0.0381
\end{verbatim}

\begin{enumerate}
\def\labelenumi{\arabic{enumi}.}
\setcounter{enumi}{4}
\tightlist
\item
  Plotting difficulty profile
\end{enumerate}

Great! We now have the difficulty for all the 15 levels in the episode.
Keep in mind that, as we measure difficulty as the probability to pass a
level in a single attempt, a lower value (a smaller probability of
winning the level) implies a higher level difficulty.

Now that we have the difficulty of the episode we should plot it. Let's
plot a line graph with the levels on the X-axis and the difficulty
(pwin) on the Y-axis. We call this plot the difficulty profile of the
episode.

\begin{Shaded}
\begin{Highlighting}[]
\CommentTok{\# Plotting the level difficulty profile}
\FunctionTok{ggplot}\NormalTok{(difficulty, }\FunctionTok{aes}\NormalTok{(}\AttributeTok{x =}\NormalTok{ level, }\AttributeTok{y =}\NormalTok{ p\_win)) }\SpecialCharTok{+}
\FunctionTok{geom\_line}\NormalTok{() }\SpecialCharTok{+}
\FunctionTok{scale\_x\_continuous}\NormalTok{(}\AttributeTok{breaks =} \DecValTok{1}\SpecialCharTok{:}\DecValTok{15}\NormalTok{) }
\end{Highlighting}
\end{Shaded}

\includegraphics{candy_files/figure-latex/unnamed-chunk-5-1.pdf} 6.
Spotting hard levels

What constitutes a hard level is subjective. However, to keep things
simple, we could define a threshold of difficulty, say 10\%, and label
levels with pwin \textless{} 10\% as hard. It's relatively easy to spot
these hard levels on the plot, but we can make the plot more friendly by
explicitly highlighting the hard levels.

\begin{Shaded}
\begin{Highlighting}[]
\CommentTok{\# Adding points and a dashed line}
\FunctionTok{ggplot}\NormalTok{(difficulty, }\FunctionTok{aes}\NormalTok{(}\AttributeTok{x =}\NormalTok{ level, }\AttributeTok{y =}\NormalTok{ p\_win)) }\SpecialCharTok{+}
\FunctionTok{geom\_line}\NormalTok{() }\SpecialCharTok{+}
\FunctionTok{scale\_x\_continuous}\NormalTok{(}\AttributeTok{breaks =} \DecValTok{1}\SpecialCharTok{:}\DecValTok{15}\NormalTok{) }\SpecialCharTok{+}
\FunctionTok{scale\_y\_continuous}\NormalTok{(}\AttributeTok{label =}\NormalTok{ scales}\SpecialCharTok{::}\NormalTok{percent) }\SpecialCharTok{+}
\FunctionTok{geom\_point}\NormalTok{() }\SpecialCharTok{+}
\FunctionTok{geom\_hline}\NormalTok{(}\AttributeTok{yintercept =} \FloatTok{0.1}\NormalTok{, }\AttributeTok{linetype =} \StringTok{"dashed"}\NormalTok{)}
\end{Highlighting}
\end{Shaded}

\includegraphics{candy_files/figure-latex/unnamed-chunk-6-1.pdf} 7.
Computing uncertainty

\begin{Shaded}
\begin{Highlighting}[]
\CommentTok{\# Computing the standard error of p\_win for each level}
\NormalTok{difficulty }\OtherTok{\textless{}{-}}\NormalTok{ difficulty }\SpecialCharTok{\%\textgreater{}\%}
    \FunctionTok{group\_by}\NormalTok{(level) }\SpecialCharTok{\%\textgreater{}\%}
    \FunctionTok{mutate}\NormalTok{(}\AttributeTok{error =} \FunctionTok{sqrt}\NormalTok{(p\_win}\SpecialCharTok{*}\NormalTok{(}\DecValTok{1}\SpecialCharTok{{-}}\NormalTok{p\_win)}\SpecialCharTok{/}\NormalTok{attempts))}

\NormalTok{difficulty}
\end{Highlighting}
\end{Shaded}

\begin{verbatim}
## # A tibble: 15 x 5
## # Groups:   level [15]
##    level attempts  wins  p_win   error
##    <dbl>    <dbl> <dbl>  <dbl>   <dbl>
##  1     1     1322   818 0.619  0.0134 
##  2     2     1285   666 0.518  0.0139 
##  3     3     1546   662 0.428  0.0126 
##  4     4     1893   705 0.372  0.0111 
##  5     5     6937   634 0.0914 0.00346
##  6     6     1591   668 0.420  0.0124 
##  7     7     4526   614 0.136  0.00509
##  8     8    15816   641 0.0405 0.00157
##  9     9     8241   670 0.0813 0.00301
## 10    10     3282   617 0.188  0.00682
## 11    11     5575   603 0.108  0.00416
## 12    12     6868   659 0.0960 0.00355
## 13    13     1327   686 0.517  0.0137 
## 14    14     2772   777 0.280  0.00853
## 15    15    30374  1157 0.0381 0.00110
\end{verbatim}

\begin{enumerate}
\def\labelenumi{\arabic{enumi}.}
\setcounter{enumi}{7}
\tightlist
\item
  Showing uncertainty
\end{enumerate}

Now that we have a measure of uncertainty for each levels' difficulty
estimate let's use error bars to show this uncertainty in the plot. We
will set the length of the error bars to one standard error. The upper
limit and the lower limit of each error bar should then be pwin + σerror
and pwin - σerror, respectively.

\begin{Shaded}
\begin{Highlighting}[]
\CommentTok{\# Adding standard error bars}
\FunctionTok{ggplot}\NormalTok{(difficulty, }\FunctionTok{aes}\NormalTok{(}\AttributeTok{x =}\NormalTok{ level, }\AttributeTok{y =}\NormalTok{ p\_win)) }\SpecialCharTok{+}
\FunctionTok{geom\_line}\NormalTok{() }\SpecialCharTok{+}
\FunctionTok{scale\_x\_continuous}\NormalTok{(}\AttributeTok{breaks =} \DecValTok{1}\SpecialCharTok{:}\DecValTok{15}\NormalTok{) }\SpecialCharTok{+}
\FunctionTok{scale\_y\_continuous}\NormalTok{(}\AttributeTok{label =}\NormalTok{ scales}\SpecialCharTok{::}\NormalTok{percent) }\SpecialCharTok{+}
\FunctionTok{geom\_point}\NormalTok{() }\SpecialCharTok{+}
\FunctionTok{geom\_hline}\NormalTok{(}\AttributeTok{yintercept =} \FloatTok{0.1}\NormalTok{, }\AttributeTok{linetype =} \StringTok{"dashed"}\NormalTok{) }\SpecialCharTok{+}
\FunctionTok{geom\_errorbar}\NormalTok{(}\FunctionTok{aes}\NormalTok{(}\AttributeTok{ymin =}\NormalTok{ p\_win }\SpecialCharTok{{-}}\NormalTok{ error, }\AttributeTok{ymax =}\NormalTok{ p\_win }\SpecialCharTok{+}\NormalTok{ error))}
\end{Highlighting}
\end{Shaded}

\includegraphics{candy_files/figure-latex/unnamed-chunk-8-1.pdf} 9. A
final metric

It looks like our difficulty estimates are pretty precise! Using this
plot, a level designer can quickly spot where the hard levels are and
also see if there seems to be too many hard levels in the episode.

One question a level designer might ask is: ``How likely is it that a
player will complete the episode without losing a single time?'' Let's
calculate this using the estimated level difficulties!

\begin{Shaded}
\begin{Highlighting}[]
\CommentTok{\# The probability of completing the episode without losing a single time}
\NormalTok{p }\OtherTok{\textless{}{-}} \FunctionTok{prod}\NormalTok{(difficulty}\SpecialCharTok{$}\NormalTok{p\_win)}

\CommentTok{\# Printing it out}
\NormalTok{p}
\end{Highlighting}
\end{Shaded}

\begin{verbatim}
## [1] 9.447141e-12
\end{verbatim}

\begin{enumerate}
\def\labelenumi{\arabic{enumi}.}
\setcounter{enumi}{9}
\tightlist
\item
  Should our level designer worry?
\end{enumerate}

Given the probability we just calculated, should our level designer
worry about that a lot of players might complete the episode in one
attempt?

No, The probability p gets printed out using scientific notation. So a
probability of 9.447-12 is the same as 0.000000000009447. That is, a
really small probability.

\end{document}
